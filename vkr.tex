\input{header.tex}
\usepackage{newfloat,caption,float}
\usepackage{newfloat}
\usepackage{subcaption}

\DeclareFloatingEnvironment[
	fileext = loe,
	listname = Examples, 
	name = Листинг,
	placement = H,
	within = none,
]{example}
\captionsetup[example]{labelfont=bf}

%\newcommand\theexample{\alph{subexample}}
%\newcommand\thesubexample{\alph{subexample}}
% \newcommand\p@subexample{\theexample}

%\DeclareCaptionSubType{#1} 
%\renewcommand\thesubexample{\theexample\alph{subexample}}
\ForEachFloatingEnvironment{\DeclareCaptionSubType*{#1}}
\captionsetup[subexample]{labelformat=simple}

\newcommand{\ReasonML}{\textsc{ReasonML}}
\newcommand{\OCaml}{\textsc{OCaml}}
\newcommand{\ocamllex}{\textsc{ocamllex}}
\newcommand{\ocamlyacc}{\textsc{ocamlyacc}}
\newcommand{\merlin}{\textsc{merlin}}

\ifx\pdfoutput\undefined
\usepackage{graphicx}
\else
\usepackage[pdftex]{graphicx}
\fi
\usepackage{float}
\usepackage{caption}
\usepackage{subcaption}


\begin{document}
\input{title.tex}
\maketitle
\setcounter{tocdepth}{2}
\tableofcontents

\section{Введение}
Reason~--- это расширение синтаксиса OCaml и новый набор инструментов, созданные в Facebook в 2016 году. Для языка был придуман C-подобный синтаксис, так же похожий и на JavaScript, что бы программистам, пишущим на этих языках, было легче адаптироваться\cite{RE}. Одной из целей создания нового языка было упрощение взаимодействия с JavaScript, добавление поддержки JSX и возможности компиляции в JavaScript\cite{WE}. В наследство от OCaml Reason-у досталась возможность компилироваться в нативный и байт-коды, а так же система типов со всеми преимуществами, такими как уменьшение количества ошибок в программах и повышение удобства поддержки кода.

Однако существует проблема у людей, уже пишущих на OCaml. Так как Reason остался достаточно похожим на него, в языках имеются одинаковые синтаксические конструкции, отличающиеся только оформлением. Из-за этого довольно сложно переключиться на новый синтаксис, и, как следствие, программист по ошибке может написать вполне корректный для OCaml участок кода, но не подходящий для Reason. В таком случае стоит указать человеку на его ошибку предупреждением и продолжить синтаксический анализ программы, корректно прочитав конструкцию из чужого языка.

Таким образом, актуальной задачей является добавление новых правил в синтаксический анализатор Reason с восстановлением от типичных ошибок, вызванных смешением языков.

\section{Постановка задачи}
\label{sec:task}
% !TeX spellcheck = ru_RU
Целью данной работы является дополнение существующего парсера Reason новыми правилами восстановления от ошибок, вызванных смешением синтаксиса Reason и OCaml.

Для её выполнения были поставлены следующие задачи:
 \begin{enumerate}
 \item Исследовать предметную область:
 \begin{enumerate} 
 	\item Рассмотреть принципы работы LR(1) анализаторов.
	\item Рассмотреть принцип работы генераторов синтаксических анализаторов.
 	\item Изучить существующую реализацию синтаксического анализатора на OCaml (Menhir).
 \end{enumerate}
 \item Реализовать дополнительные правила восстановления от ошибок.
 \item Оценить качество восстановления от ошибок.
 
 \end{enumerate}

\section{Обзор}
\label{sec:relatedworks}

\subsection{Синтаксические анализаторы}
Синтаксический анализ или парсинг (parsing)~--- процесс сопоставления последовательности токенов (лексем) дерева разбора, называемого абстрактным синтаксическим деревом (AST).

Синтаксический анализатор или парсер (parser)~--- это программа, выполняющая синтаксический анализ.

Так же от синтаксического анализатора ожидаются сообщения обо всех выявленных ошибках, причем достаточно внятных и полных, а кроме того, умение обрабатывать обычные, часто встречающиеся ошибки и продолжать работу с оставшейся частью программы.
Имеется три основных типа синтаксических анализаторов: универсальные, восходящие и нисходящие\cite{DB}. Мы остановимся на восходящем анализе. Восходящие синтаксические анализаторы строят дерево разбора начиная с листьев (снизу) и идя к корню (вверх). Поток символов сканируется последовательно --- слева направо.

Здесь будет рассмотрен общий вид восходящего анализа типа <<перенос/свёртка>> (shift-reduce). При анализе типа перенос-свёртка для хранения символов грамматики используется стек, а для хранения остающейся непроанализированной части входной строки~--- входной буфер. На каждом шаге свёртки (reduction) определенная подстрока, соответствующая правой части продукции, заменяется на нетерминал, являющийся левой частью продукции. Основа~--- это подстрока, которая соответствует телу продукции и свертка которой представляет собой один шаг правого порождения в обратном порядке. Использование стека в анализаторе объясняется тем важным фактом, что {\it основа} всегда находится на вершине стека и никогда~--- внутри него.

LR(k)-грамматики~--- это грамматики, для которых может быть построен синтаксический анализатор, работающий по принципу переноса-свёртки.

Наиболее эффективные восходящие методы работают только с подклассами грамматик, однако некоторые из этих классов, такие как LR(k) грамматики, достаточно выразительны для описания большинства синтаксических конструкций языков программирования.
L~--- здесь означает сканирование входного потока слева направо, R~--- построение правого порождения в обратном порядке, а k --- количество предпросматриваемых символов входного потока, необходимых для принятия решения. 

Одними из самых эффективных синтаксических анализаторов являются LR(1)\cite{LRSpeed}. LR(1)-анализатор состоит из входного буфера, выхода, стека, программы-драйвера и таблицы синтаксического анализа. Программа-драйвер одинакова для всех LR-анализаторов, от одного анализатора к другому меняются таблицы синтаксического анализа. Программа ситаксического анализатора по одному считывает символы из входного буфера. После прочтения очередного символа она обращается к управляющей таблице и совершает соответствующее действие. Процесс чтения продолжается, пока входная цепочка не закончится.

\subsection{Menhir}
Построение анализаторов LR(1) вручную достаточно трудоемкий процесс, поэтому чаще всего пользуются генераторами синтаксических анализаторов. 

Генератор синтаксических анализаторов --- это программа, которая по спецификации грамматики строит синтаксический анализатор. Одним из таких генераторов и является Menhir~\cite{ME}. Он был выбран командой Reason, как лучшая, по сравнению с ocamlyacc, версия генераторов синтаксических анализаторов.



\subsection{Примеры ошибок}

Здесь будут приведены примеры синтаксических конструкций, в которых часто встречаются ошибки, в формате сравнения кода на OCaml и Reason соответственно.

\hfill


\begin{example}
	\begin{subexample}[b]{\textwidth}
		\lstinputlisting[frame=single,language=Caml]{examplePatternMatching.ml}
		\caption{}
	\end{subexample}
	\hfill
	
	\begin{subexample}[b]{\textwidth}
		\lstinputlisting[frame=single, language=Caml]{examplePatternMatching.re}
		\caption{}
	\end{subexample}
\caption{}\label{l1}
\end{example}

На листинге \ref{l1} представлена конструкция сопоставления с образцом (pattern matching). Программист может перепутать оформление стрелок (-> и =>), выбрать ключевые слова из другого языка ({\it match - with} и {\it switch(-)}) или забыть поставить фигурные скобки.


\hfill


\begin{example}
%\begin{multicols}{2}
	\begin{subexample}{0.5\textwidth}
		\lstinputlisting[frame=single, language=Caml]{forloop.ml}
		\caption{}
	\end{subexample}
	\begin{subexample}{0.5\textwidth}
		\lstinputlisting[frame=single, language=Caml]{forloop.re}
		\caption{}
	\end{subexample}
%\end{multicols}
\caption{}\label{l2}
\end{example}

На листинге \ref{l2} представлен стандартный цикл. Человек может вместо фигурных скобок заключать тело цикла в {\it do} и {\it done}, неправильно инициализировать переменную ({\it i = 1} и {\it i in 1}) или забыть фигурные скобки после {\it for}.


\hfill


\begin{example}
%\begin{multicols}{2}
	\begin{subexample}{0.5\textwidth}
		\lstinputlisting[frame=single, language=Caml]{record.ml}
		\caption{}
	\end{subexample}
	\begin{subexample}{0.5\textwidth}
		\lstinputlisting[frame=single, language=Caml]{record.re}
		\caption{}
	\end{subexample}
%\end{multicols}
\caption{}\label{l3}
\end{example}

На листинге \ref{l3} приведено описание типа, называемого записью. В первом случае поля отделяются точкой с запятой, во втором случае просто запятой.

\subsection{Merlin}

Проект \merlin{}\cite{mer} является реализацией сервера для анализа языка \OCaml{} (language server). Чаще всего он подключается к редакторам текста с помощью различных расширений этих редакторов, чтобы разработчик мог использовать редактор как современную интегрированную среду разработки (IDE), т.\,е. пользоваться переходом к определению объявленных имен, автодополнением и т.\,п. На данный момент в исходном коде \ReasonML{} имеется расширение для \merlin{}, которое позволяет использовать \merlin{} не только для редактирования кода на \OCaml{}, но и на \ReasonML{}.

В контексте данной работы выглядит интересной задача тесной интеграции синтаксического анализатора с проектом \merlin{}, а именно добавление возможности быстрого исправления кода, написанного со смешением двух языков, на правильный код на \ReasonML{}.


%\subsection{Реализация дополнительных правил}



\section{Результаты}
В силу проблем возникающих при смешении грамматик языков и возникающих при этом shift/reduce или reduce/reduce конфликтов некоторые правила для конструкций реализованы не в полном объеме.
\begin{itemize}

\item Добавлено ключевое слово {\it match} и теперь оно не может использоваться в качестве идентификатора.

\item Можно использовать другой стиль стрелок («->» вместо «=>») в pattern matching, но в таком случае не работает {\it when} (одна из дополнительных функций).

\item Можно писать pattern matching в стиле \OCaml{}.%, но тогда после стрелки может идти только единичное выражение, а не их последовательность.%\footnote[1]{Однако эта проблема легко решается заключением последовательности в \{\}. }

\item Можно писать условие и тело цикла в стиле \OCaml{}.

\item Можно использовать разделитель <<;>> вместо <<,>> в объявлении record.

\item Можно объявлять модули в стиле \OCaml{}.

\item Можно писать {\it if} с {\it then}.

\end{itemize}

Так же добавлены предупреждения (warning) для всех синтаксических конструкций перечисленных выше. И реализовано быстрое исправление кода (<<Quick fix>>) в OCaml-LSP, которое заменяет {\it match with} и {\it struct end} на подходящие языку Reason ключевые слова.
\begin{figure}[h]
\begin{subfigure}{0.5\textwidth}
	\includegraphics[width=\linewidth]{screenshots/05.png}
\end{subfigure}
\begin{subfigure}{0.5\textwidth}
	\includegraphics[width=\linewidth]{screenshots/06.png}
\end{subfigure}
\caption{Примеры}
\end{figure}

\newpage
Для симметричной задачи, для OCaml реализованы предупреждения в OCaml-LSP для следующих конструкций из Reason: определение модуля, pattern matching (при этом стрелки должны быть из OCaml), запятые в record. Для этих конструкций реализованы соответствующие быстрые исправления в OCaml-LSP:
\begin{itemize}

\item Замена {\it \{ \}} на {\it struct end}.

\item Замена {\it switch \{ \}} на {\it match with}.

\item Замена всех <<,>> на <<;>> в объявлении record.

\end{itemize}

\begin{figure}[h]
\begin{subfigure}{0.5\textwidth}
	\includegraphics[width=\linewidth]{screenshots/01.png}
\end{subfigure}
\begin{subfigure}{0.5\textwidth}
	\includegraphics[width=\linewidth]{screenshots/02.png}
\end{subfigure}
\newline
\begin{subfigure}{0.5\textwidth}
	\includegraphics[width=\linewidth]{screenshots/03.png}
\end{subfigure}
\begin{subfigure}{0.5\textwidth}
	\includegraphics[width=\linewidth]{screenshots/04.png}
\end{subfigure}
\caption{Исправление кода OCaml}
	\label{fix}%fig3
\end{figure}

Результаты проделанной работы можно наблюдать в соответсвующих репозиториях. Ссылки на списки коммитов Reason\footnote[1]{ \url{https://github.com/pereb4ik/reason/commits?author=pereb4ik} },
OCaml-LSP\footnote[2]{ \url{https://github.com/pereb4ik/ocaml-lsp/commits?author=pereb4ik} },
merlin\footnote[3]{ \url{https://github.com/pereb4ik/merlin/commits/lsp?author=pereb4ik} }.
\newpage

\subsection{Установка}
Необходимая версия OCaml~--- 4.13.1/4.13.
 
Reason устанавливается через:
% ":" not italic, i don't know
\lstset{emph={https}, emphstyle={\itshape \small} }
\lstset{basicstyle=\ttfamily}
\begin{example}
\begin{lstlisting}[frame=single,language=C]
opam pin add reason https://github.com/pereb4ik/reason.git
\end{lstlisting}
\caption{}
\end{example}
При необходимости так же rtop:
\begin{example}
\begin{lstlisting}[frame=single,language=C]
opam pin add rtop https://github.com/pereb4ik/reason.git
\end{lstlisting}
\caption{}
\end{example}
Для тестирования функций IDE, умной замены кода <<Quick fix>> понадобится Visual Studio Code (хотя ocaml-lsp можно использовать с другими редакторами) с расширением OCaml Platform\footnote[4]{ \url{https://marketplace.visualstudio.com/items?itemName=ocamllabs.ocaml-platform} } от OCaml Labs. И пакет ocaml-lsp <<нашей>> версии:
\begin{example}
\begin{lstlisting}[frame=single,
language=C,
% fix stupid bug with bad "-" in copying from pdf
literate={-}{-}1
]
opam pin add ocaml-lsp-server https://github.com/pereb4ik/ocaml-lsp.git
\end{lstlisting}
\caption{}
\end{example}

\section{Заключение}
В рамках данной работы было проведено подготовительное исследование возможности расширения синтаксического анализатора \ReasonML{} правилами для языка \OCaml{}. В ходе исследования были выполнены и начата работа над следующими задачами.


\begin{enumerate}
\item[\checkmark] Исследовать предметную область:
\begin{enumerate} 
\item[\checkmark] Рассмотреть принципы работы LR(1) анализаторов.
\item[\checkmark] Рассмотреть принцип работы генераторов синтаксических анализаторов.
\item[\checkmark] Изучить существующую реализацию синтаксического анализатора на \OCaml{} (Menhir).
\end{enumerate}
\item[\checkmark] Реализовать дополнительные правила восстановления от ошибок.
\item[$\times$] Оценить качество восстановления от ошибок.
\end{enumerate}

\noindent В процессе работы возникали опасения того, что лексический анализатор \ReasonML{} на основе \ocamllex{} потребует доработки. Однако, большинство полезных лексем там уже поддерживается и добавление новых не должно вызвать серьёзных усилий. 

В оставшийся срок работы планируется реализовать изменения в синтаксическом анализаторе \ReasonML{} для улучшения поддержки синтаксических конструкций языка \OCaml{}. При удачном стечении обстоятельств планируется доработать \merlin{} и расширение для редактора \textsc{VsCode} поддержкой возможностей автоматического исправления кода, написанного с использованием смешения языков.

В планах на будущее выглядит интересным не только улучшение синтаксического анализатора \ReasonML{} для облегчения жизни программистам на \OCaml{}, но и наоборот: доработка синтаксического анализатора \OCaml{} в проекте \merlin{} для более удобной миграции программистов на \ReasonML{} в сообщество языка \OCaml{}.


% \nocite{*}
\setmonofont[Mapping=tex-text]{CMU Typewriter Text}
\bibliographystyle{ugost2008ls}
\bibliography{vkr}
\end{document}

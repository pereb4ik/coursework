Reason~--- это расширение синтаксиса OCaml и новый набор инструментов, созданные в Facebook в 2016 году. Для языка был придуман C-подобный синтаксис, так же похожий и на JavaScript, что бы программистам, пишущим на этих языках, было легче адаптироваться\cite{RE}. Одной из целей создания нового языка было упрощение взаимодействия с JavaScript, добавление поддержки JSX и возможности компиляции в JavaScript\cite{WE}. В наследство от OCaml Reason-у досталась возможность компилироваться в нативный и байт-коды, а так же система типов со всеми преимуществами, такими как уменьшение количества ошибок в программах и повышение удобства поддержки кода.

Однако существует проблема у людей, уже пишущих на OCaml. Так как Reason остался достаточно похожим на него, в языках имеются одинаковые синтаксические конструкции, отличающиеся только оформлением. Из-за этого довольно сложно переключиться на новый синтаксис, и, как следствие, программист по ошибке может написать вполне корректный для OCaml участок кода, но не подходящий для Reason. В таком случае стоит указать человеку на его ошибку предупреждением и продолжить синтаксический анализ программы, корректно прочитав конструкцию из чужого языка.

Таким образом, актуальной задачей является добавление новых правил в синтаксический анализатор Reason с восстановлением от типичных ошибок, вызванных смешением языков.
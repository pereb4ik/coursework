В рамках данной работы было проведено подготовительное исследование возможности расширения синтаксического анализатора \ReasonML{} правилами для языка \OCaml{}. В ходе исследования были выполнены и начата работа над следующими задачами.


\begin{enumerate}
\item[\checkmark] Исследовать предметную область:
\begin{enumerate} 
\item[\checkmark] Рассмотреть принципы работы LR(1) анализаторов.
\item[\checkmark] Рассмотреть принцип работы генераторов синтаксических анализаторов.
\item[\checkmark] Изучить существующую реализацию синтаксического анализатора на \OCaml{} (Menhir).
\end{enumerate}
\item[\checkmark] Реализовать дополнительные правила восстановления от ошибок.
\item[\checkmark] Реализовать в merlin и Reason предупреждения о смешении кода разных языков.
\item[\checkmark] Реализовать в ocaml-lsp автоматическое исправление кода OCaml, добавленного в Reason.
\item[\checkmark] Реализовать в ocaml-lsp автоматическое исправление кода Reason, добавленного в OCaml.
\end{enumerate}

\noindent В процессе работы возникали опасения того, что лексический анализатор \ReasonML{} на основе \ocamllex{} потребует доработки. Однако, большинство полезных лексем там уже поддерживается и добавление новых не должно вызвать серьёзных усилий. 

%В оставшийся срок работы планируется реализовать изменения в синтаксическом анализаторе \ReasonML{} для улучшения поддержки синтаксических конструкций языка \OCaml{}. При удачном стечении обстоятельств планируется доработать \merlin{} и расширение для редактора \textsc{VsCode} поддержкой возможностей автоматического исправления кода, написанного с использованием смешения языков.

%В планах на будущее выглядит интересным не только улучшение синтаксического анализатора \ReasonML{} для облегчения жизни программистам на \OCaml{}, но и наоборот: доработка синтаксического анализатора \OCaml{} в проекте \merlin{} для более удобной миграции программистов на \ReasonML{} в сообщество языка \OCaml{}.